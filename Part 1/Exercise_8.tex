% Options for packages loaded elsewhere
\PassOptionsToPackage{unicode}{hyperref}
\PassOptionsToPackage{hyphens}{url}
%
\documentclass[
]{article}
\usepackage{amsmath,amssymb}
\usepackage{lmodern}
\usepackage{iftex}
\ifPDFTeX
  \usepackage[T1]{fontenc}
  \usepackage[utf8]{inputenc}
  \usepackage{textcomp} % provide euro and other symbols
\else % if luatex or xetex
  \usepackage{unicode-math}
  \defaultfontfeatures{Scale=MatchLowercase}
  \defaultfontfeatures[\rmfamily]{Ligatures=TeX,Scale=1}
\fi
% Use upquote if available, for straight quotes in verbatim environments
\IfFileExists{upquote.sty}{\usepackage{upquote}}{}
\IfFileExists{microtype.sty}{% use microtype if available
  \usepackage[]{microtype}
  \UseMicrotypeSet[protrusion]{basicmath} % disable protrusion for tt fonts
}{}
\makeatletter
\@ifundefined{KOMAClassName}{% if non-KOMA class
  \IfFileExists{parskip.sty}{%
    \usepackage{parskip}
  }{% else
    \setlength{\parindent}{0pt}
    \setlength{\parskip}{6pt plus 2pt minus 1pt}}
}{% if KOMA class
  \KOMAoptions{parskip=half}}
\makeatother
\usepackage{xcolor}
\usepackage[margin=1in]{geometry}
\usepackage{color}
\usepackage{fancyvrb}
\newcommand{\VerbBar}{|}
\newcommand{\VERB}{\Verb[commandchars=\\\{\}]}
\DefineVerbatimEnvironment{Highlighting}{Verbatim}{commandchars=\\\{\}}
% Add ',fontsize=\small' for more characters per line
\usepackage{framed}
\definecolor{shadecolor}{RGB}{248,248,248}
\newenvironment{Shaded}{\begin{snugshade}}{\end{snugshade}}
\newcommand{\AlertTok}[1]{\textcolor[rgb]{0.94,0.16,0.16}{#1}}
\newcommand{\AnnotationTok}[1]{\textcolor[rgb]{0.56,0.35,0.01}{\textbf{\textit{#1}}}}
\newcommand{\AttributeTok}[1]{\textcolor[rgb]{0.77,0.63,0.00}{#1}}
\newcommand{\BaseNTok}[1]{\textcolor[rgb]{0.00,0.00,0.81}{#1}}
\newcommand{\BuiltInTok}[1]{#1}
\newcommand{\CharTok}[1]{\textcolor[rgb]{0.31,0.60,0.02}{#1}}
\newcommand{\CommentTok}[1]{\textcolor[rgb]{0.56,0.35,0.01}{\textit{#1}}}
\newcommand{\CommentVarTok}[1]{\textcolor[rgb]{0.56,0.35,0.01}{\textbf{\textit{#1}}}}
\newcommand{\ConstantTok}[1]{\textcolor[rgb]{0.00,0.00,0.00}{#1}}
\newcommand{\ControlFlowTok}[1]{\textcolor[rgb]{0.13,0.29,0.53}{\textbf{#1}}}
\newcommand{\DataTypeTok}[1]{\textcolor[rgb]{0.13,0.29,0.53}{#1}}
\newcommand{\DecValTok}[1]{\textcolor[rgb]{0.00,0.00,0.81}{#1}}
\newcommand{\DocumentationTok}[1]{\textcolor[rgb]{0.56,0.35,0.01}{\textbf{\textit{#1}}}}
\newcommand{\ErrorTok}[1]{\textcolor[rgb]{0.64,0.00,0.00}{\textbf{#1}}}
\newcommand{\ExtensionTok}[1]{#1}
\newcommand{\FloatTok}[1]{\textcolor[rgb]{0.00,0.00,0.81}{#1}}
\newcommand{\FunctionTok}[1]{\textcolor[rgb]{0.00,0.00,0.00}{#1}}
\newcommand{\ImportTok}[1]{#1}
\newcommand{\InformationTok}[1]{\textcolor[rgb]{0.56,0.35,0.01}{\textbf{\textit{#1}}}}
\newcommand{\KeywordTok}[1]{\textcolor[rgb]{0.13,0.29,0.53}{\textbf{#1}}}
\newcommand{\NormalTok}[1]{#1}
\newcommand{\OperatorTok}[1]{\textcolor[rgb]{0.81,0.36,0.00}{\textbf{#1}}}
\newcommand{\OtherTok}[1]{\textcolor[rgb]{0.56,0.35,0.01}{#1}}
\newcommand{\PreprocessorTok}[1]{\textcolor[rgb]{0.56,0.35,0.01}{\textit{#1}}}
\newcommand{\RegionMarkerTok}[1]{#1}
\newcommand{\SpecialCharTok}[1]{\textcolor[rgb]{0.00,0.00,0.00}{#1}}
\newcommand{\SpecialStringTok}[1]{\textcolor[rgb]{0.31,0.60,0.02}{#1}}
\newcommand{\StringTok}[1]{\textcolor[rgb]{0.31,0.60,0.02}{#1}}
\newcommand{\VariableTok}[1]{\textcolor[rgb]{0.00,0.00,0.00}{#1}}
\newcommand{\VerbatimStringTok}[1]{\textcolor[rgb]{0.31,0.60,0.02}{#1}}
\newcommand{\WarningTok}[1]{\textcolor[rgb]{0.56,0.35,0.01}{\textbf{\textit{#1}}}}
\usepackage{graphicx}
\makeatletter
\def\maxwidth{\ifdim\Gin@nat@width>\linewidth\linewidth\else\Gin@nat@width\fi}
\def\maxheight{\ifdim\Gin@nat@height>\textheight\textheight\else\Gin@nat@height\fi}
\makeatother
% Scale images if necessary, so that they will not overflow the page
% margins by default, and it is still possible to overwrite the defaults
% using explicit options in \includegraphics[width, height, ...]{}
\setkeys{Gin}{width=\maxwidth,height=\maxheight,keepaspectratio}
% Set default figure placement to htbp
\makeatletter
\def\fps@figure{htbp}
\makeatother
\setlength{\emergencystretch}{3em} % prevent overfull lines
\providecommand{\tightlist}{%
  \setlength{\itemsep}{0pt}\setlength{\parskip}{0pt}}
\setcounter{secnumdepth}{-\maxdimen} % remove section numbering
\ifLuaTeX
  \usepackage{selnolig}  % disable illegal ligatures
\fi
\IfFileExists{bookmark.sty}{\usepackage{bookmark}}{\usepackage{hyperref}}
\IfFileExists{xurl.sty}{\usepackage{xurl}}{} % add URL line breaks if available
\urlstyle{same} % disable monospaced font for URLs
\hypersetup{
  hidelinks,
  pdfcreator={LaTeX via pandoc}}

\author{}
\date{\vspace{-2.5em}}

\begin{document}

\hypertarget{exercise-8}{%
\section{Exercise 8}\label{exercise-8}}

\begin{Shaded}
\begin{Highlighting}[]
\CommentTok{\#Set working directory}
\FunctionTok{rm}\NormalTok{(}\AttributeTok{list=}\FunctionTok{ls}\NormalTok{())}
\FunctionTok{setwd}\NormalTok{(}\StringTok{"C:}\SpecialCharTok{\textbackslash{}\textbackslash{}}\StringTok{Users}\SpecialCharTok{\textbackslash{}\textbackslash{}}\StringTok{Remo\_}\SpecialCharTok{\textbackslash{}\textbackslash{}}\StringTok{OneDrive}\SpecialCharTok{\textbackslash{}\textbackslash{}}\StringTok{Dokumente}\SpecialCharTok{\textbackslash{}\textbackslash{}}\StringTok{BigDataSeminar"}\NormalTok{)}
\end{Highlighting}
\end{Shaded}

\begin{Shaded}
\begin{Highlighting}[]
\CommentTok{\#install the packages}
\CommentTok{\#install.packages("data.table")}
\CommentTok{\#install.packages("nnet")}
\CommentTok{\#install.packages("ISLR")}
\CommentTok{\#install.packages("ggplot2")}
\CommentTok{\#install.packages("reshape")}
\end{Highlighting}
\end{Shaded}

\begin{Shaded}
\begin{Highlighting}[]
\CommentTok{\#Load the packages}
\FunctionTok{library}\NormalTok{(data.table)}
\end{Highlighting}
\end{Shaded}

\begin{verbatim}
## Warning: package 'data.table' was built under R version 4.1.3
\end{verbatim}

\begin{Shaded}
\begin{Highlighting}[]
\FunctionTok{library}\NormalTok{(nnet)}
\end{Highlighting}
\end{Shaded}

\begin{verbatim}
## Warning: package 'nnet' was built under R version 4.1.3
\end{verbatim}

\begin{Shaded}
\begin{Highlighting}[]
\FunctionTok{library}\NormalTok{(ISLR)}
\end{Highlighting}
\end{Shaded}

\begin{verbatim}
## Warning: package 'ISLR' was built under R version 4.1.3
\end{verbatim}

\begin{Shaded}
\begin{Highlighting}[]
\FunctionTok{library}\NormalTok{(ggplot2)}
\end{Highlighting}
\end{Shaded}

\begin{verbatim}
## Warning: package 'ggplot2' was built under R version 4.1.3
\end{verbatim}

\begin{Shaded}
\begin{Highlighting}[]
\FunctionTok{library}\NormalTok{(devtools)}
\end{Highlighting}
\end{Shaded}

\begin{verbatim}
## Warning: package 'devtools' was built under R version 4.1.3
\end{verbatim}

\begin{verbatim}
## Loading required package: usethis
\end{verbatim}

\begin{verbatim}
## Warning: package 'usethis' was built under R version 4.1.3
\end{verbatim}

\begin{Shaded}
\begin{Highlighting}[]
\FunctionTok{library}\NormalTok{(reshape)}
\end{Highlighting}
\end{Shaded}

\begin{verbatim}
## Warning: package 'reshape' was built under R version 4.1.3
\end{verbatim}

\begin{verbatim}
## 
## Attaching package: 'reshape'
\end{verbatim}

\begin{verbatim}
## The following object is masked from 'package:data.table':
## 
##     melt
\end{verbatim}

\begin{Shaded}
\begin{Highlighting}[]
\CommentTok{\#import the plotting function from Github}
\FunctionTok{source\_url}\NormalTok{(}\StringTok{\textquotesingle{}https://gist.githubusercontent.com/fawda123/7471137/raw/466c1474d0a505ff044412703516c34f1a4684a5/nnet\_plot\_update.r\textquotesingle{}}\NormalTok{)}
\end{Highlighting}
\end{Shaded}

\begin{verbatim}
## i SHA-1 hash of file is "74c80bd5ddbc17ab3ae5ece9c0ed9beb612e87ef"
\end{verbatim}

\hypertarget{a}{%
\subsection{(a)}\label{a}}

Create the data set on which we want to do a simple regression. Set the
seed to 42, generate 200 random points between -10 and 10 and store them
in a vector named X. Then, create a vector named Y containing the value
of sin(x).

\begin{Shaded}
\begin{Highlighting}[]
\CommentTok{\#Dataset creation}

\FunctionTok{set.seed}\NormalTok{(}\DecValTok{42}\NormalTok{)}
\NormalTok{X }\OtherTok{\textless{}{-}} \FunctionTok{runif}\NormalTok{(}\DecValTok{200}\NormalTok{, }\SpecialCharTok{{-}}\DecValTok{10}\NormalTok{, }\DecValTok{10}\NormalTok{)}
\NormalTok{Y }\OtherTok{\textless{}{-}} \FunctionTok{sin}\NormalTok{(X)}

\NormalTok{data }\OtherTok{\textless{}{-}} \FunctionTok{data.frame}\NormalTok{(X, Y)}

\CommentTok{\#Plot the created dataset}

\FunctionTok{ggplot}\NormalTok{()}\SpecialCharTok{+}
  \FunctionTok{geom\_point}\NormalTok{(}\AttributeTok{data=}\NormalTok{data, }\FunctionTok{aes}\NormalTok{(}\AttributeTok{x=}\NormalTok{X, }\AttributeTok{y=}\NormalTok{Y))}
\end{Highlighting}
\end{Shaded}

\includegraphics{Exercise_8_files/figure-latex/unnamed-chunk-4-1.pdf}

\hypertarget{b}{%
\subsection{(b)}\label{b}}

Use a feed-forward neural network and the logistic activation which are
the defaults for the package nnet. We take one number as input of our
neural network and we want one number as the output so the size of the
input and output layer are both of one. For the hidden layer, we'll
start with three neurons. It's good practice to randomize the initial
weights, so create a vector of 10 random values, picked in the interval
{[}-1,1{]}.

\begin{Shaded}
\begin{Highlighting}[]
\CommentTok{\#Initialize weights}
\NormalTok{weights10 }\OtherTok{\textless{}{-}} \FunctionTok{runif}\NormalTok{(}\DecValTok{10}\NormalTok{, }\AttributeTok{min=}\SpecialCharTok{{-}}\DecValTok{1}\NormalTok{, }\AttributeTok{max=}\DecValTok{1}\NormalTok{)}
\end{Highlighting}
\end{Shaded}

\hypertarget{c}{%
\subsection{(c)}\label{c}}

Split data to a training set containing 75\% of the values in your
initial data set and a test set containing the rest of your data.

\begin{Shaded}
\begin{Highlighting}[]
\CommentTok{\# define the Split}
\NormalTok{split }\OtherTok{=} \FloatTok{0.75}
\NormalTok{n\_train }\OtherTok{\textless{}{-}} \FunctionTok{nrow}\NormalTok{(data) }\SpecialCharTok{*}\NormalTok{ split}

\CommentTok{\# sample row indices}

\NormalTok{train\_indices }\OtherTok{\textless{}{-}} \FunctionTok{sample}\NormalTok{(}\FunctionTok{seq\_len}\NormalTok{(}\FunctionTok{nrow}\NormalTok{(data)), }\AttributeTok{size =}\NormalTok{ n\_train, }\AttributeTok{replace =} \ConstantTok{FALSE}\NormalTok{)}

\CommentTok{\# Split the data}
\NormalTok{train }\OtherTok{\textless{}{-}}\NormalTok{ data[train\_indices,]}
\NormalTok{test }\OtherTok{\textless{}{-}}\NormalTok{ data[}\SpecialCharTok{{-}}\NormalTok{train\_indices,]}
\end{Highlighting}
\end{Shaded}

\hypertarget{d}{%
\subsection{(d)}\label{d}}

Load the nnet package and use the function of the same name to create
your model. Pass your weights via the Wts argument and set the maxit
argument to 50. We want to fit a function which can have for output
multiple possible values. To do so, set the linout argument to true.
Finally, take the time to look at the structure of your model.

\begin{Shaded}
\begin{Highlighting}[]
\CommentTok{\# Create Model}
\NormalTok{model1 }\OtherTok{\textless{}{-}} \FunctionTok{nnet}\NormalTok{(Y }\SpecialCharTok{\textasciitilde{}}\NormalTok{ X, }\AttributeTok{data =}\NormalTok{ train, }\AttributeTok{size =} \DecValTok{3}\NormalTok{, }\AttributeTok{Wts =}\NormalTok{ weights10, }\AttributeTok{linout =} \ConstantTok{TRUE}\NormalTok{, }\AttributeTok{maxit=}\DecValTok{50}\NormalTok{)}
\end{Highlighting}
\end{Shaded}

\begin{verbatim}
## # weights:  10
## initial  value 111.178747 
## iter  10 value 71.469217
## iter  20 value 65.058700
## iter  30 value 33.986905
## iter  40 value 32.960766
## iter  50 value 32.659436
## final  value 32.659436 
## stopped after 50 iterations
\end{verbatim}

\begin{Shaded}
\begin{Highlighting}[]
\CommentTok{\#plot the model}
\FunctionTok{plot.nnet}\NormalTok{(model1)}
\end{Highlighting}
\end{Shaded}

\begin{verbatim}
## Loading required package: scales
\end{verbatim}

\begin{verbatim}
## Warning: package 'scales' was built under R version 4.1.3
\end{verbatim}

\includegraphics{Exercise_8_files/figure-latex/unnamed-chunk-7-1.pdf}

\hypertarget{e}{%
\subsection{(e)}\label{e}}

Predict the output for the test set and compute the RMSE of your
predictions. Plot the function sin(x) and then plot your predictions.

\begin{Shaded}
\begin{Highlighting}[]
\NormalTok{test}\SpecialCharTok{$}\NormalTok{pred1 }\OtherTok{\textless{}{-}} \FunctionTok{predict}\NormalTok{(model1, }\AttributeTok{newdata =}\NormalTok{ test)}

\CommentTok{\# RMSE}

\NormalTok{rmse }\OtherTok{\textless{}{-}} \FunctionTok{sqrt}\NormalTok{(}\FunctionTok{mean}\NormalTok{((test}\SpecialCharTok{$}\NormalTok{pred1 }\SpecialCharTok{{-}}\NormalTok{ test}\SpecialCharTok{$}\NormalTok{Y)}\SpecialCharTok{\^{}}\DecValTok{2}\NormalTok{))}
\FunctionTok{cat}\NormalTok{(}\StringTok{"RMSE:"}\NormalTok{, rmse, }\StringTok{"}\SpecialCharTok{\textbackslash{}n}\StringTok{"}\NormalTok{)}
\end{Highlighting}
\end{Shaded}

\begin{verbatim}
## RMSE: 0.4687425
\end{verbatim}

\hypertarget{f}{%
\subsection{(f)}\label{f}}

The number of neurons in the hidden layer, as well as the number of
hidden layer used, has a great influence on the effectiveness of your
model. Repeat the exercises (c) to (e), but this time use a hidden layer
with seven neurons and initiate randomly 22 weights.

\begin{Shaded}
\begin{Highlighting}[]
\CommentTok{\#Initialize weights}
\NormalTok{weights22 }\OtherTok{=} \FunctionTok{runif}\NormalTok{(}\DecValTok{22}\NormalTok{, }\SpecialCharTok{{-}}\DecValTok{1}\NormalTok{, }\DecValTok{1}\NormalTok{)}

\CommentTok{\# use nnet to create neural network}
\NormalTok{model2 }\OtherTok{\textless{}{-}} \FunctionTok{nnet}\NormalTok{(Y }\SpecialCharTok{\textasciitilde{}}\NormalTok{ X, }\AttributeTok{data =}\NormalTok{ train, }\AttributeTok{size =} \DecValTok{7}\NormalTok{, }\AttributeTok{Wts =}\NormalTok{ weights22, }\AttributeTok{linout =} \ConstantTok{TRUE}\NormalTok{, }\AttributeTok{maxit=}\DecValTok{50}\NormalTok{)}
\end{Highlighting}
\end{Shaded}

\begin{verbatim}
## # weights:  22
## initial  value 116.080419 
## iter  10 value 71.906680
## iter  20 value 47.571657
## iter  30 value 36.376527
## iter  40 value 27.212612
## iter  50 value 24.837070
## final  value 24.837070 
## stopped after 50 iterations
\end{verbatim}

\begin{Shaded}
\begin{Highlighting}[]
\CommentTok{\# Predict}
\NormalTok{test}\SpecialCharTok{$}\NormalTok{pred2 }\OtherTok{\textless{}{-}} \FunctionTok{predict}\NormalTok{(model2, }\AttributeTok{newdata =}\NormalTok{ test)}

\CommentTok{\# RMSE}
\NormalTok{rmse }\OtherTok{\textless{}{-}} \FunctionTok{sqrt}\NormalTok{(}\FunctionTok{mean}\NormalTok{((test}\SpecialCharTok{$}\NormalTok{pred2 }\SpecialCharTok{{-}}\NormalTok{ test}\SpecialCharTok{$}\NormalTok{Y)}\SpecialCharTok{\^{}}\DecValTok{2}\NormalTok{))}
\FunctionTok{cat}\NormalTok{(}\StringTok{"RMSE:"}\NormalTok{, rmse, }\StringTok{"}\SpecialCharTok{\textbackslash{}n}\StringTok{"}\NormalTok{)}
\end{Highlighting}
\end{Shaded}

\begin{verbatim}
## RMSE: 0.3955487
\end{verbatim}

\begin{Shaded}
\begin{Highlighting}[]
\CommentTok{\#plot the model}
\FunctionTok{plot.nnet}\NormalTok{(model2)}
\end{Highlighting}
\end{Shaded}

\includegraphics{Exercise_8_files/figure-latex/unnamed-chunk-9-1.pdf}

\hypertarget{further}{%
\subsection{Further}\label{further}}

Running our own model with more iterations

\begin{Shaded}
\begin{Highlighting}[]
\CommentTok{\# use nnet to create neural network}
\NormalTok{model3 }\OtherTok{\textless{}{-}} \FunctionTok{nnet}\NormalTok{(Y }\SpecialCharTok{\textasciitilde{}}\NormalTok{ X, }\AttributeTok{data =}\NormalTok{ train, }\AttributeTok{size =} \DecValTok{7}\NormalTok{, }\AttributeTok{Wts =}\NormalTok{ weights22, }\AttributeTok{linout =} \ConstantTok{TRUE}\NormalTok{, }\AttributeTok{maxit=}\DecValTok{100}\NormalTok{)}
\end{Highlighting}
\end{Shaded}

\begin{verbatim}
## # weights:  22
## initial  value 116.080419 
## iter  10 value 71.906680
## iter  20 value 47.571657
## iter  30 value 36.376527
## iter  40 value 27.212612
## iter  50 value 24.837070
## iter  60 value 16.857453
## iter  70 value 6.298826
## iter  80 value 2.559546
## iter  90 value 1.795072
## iter 100 value 1.414669
## final  value 1.414669 
## stopped after 100 iterations
\end{verbatim}

\begin{Shaded}
\begin{Highlighting}[]
\CommentTok{\# Predict}
\NormalTok{test}\SpecialCharTok{$}\NormalTok{pred3 }\OtherTok{\textless{}{-}} \FunctionTok{predict}\NormalTok{(model3, }\AttributeTok{newdata =}\NormalTok{ test)}

\CommentTok{\# RMSE}
\NormalTok{rmse }\OtherTok{\textless{}{-}} \FunctionTok{sqrt}\NormalTok{(}\FunctionTok{mean}\NormalTok{((test}\SpecialCharTok{$}\NormalTok{pred3 }\SpecialCharTok{{-}}\NormalTok{ test}\SpecialCharTok{$}\NormalTok{Y)}\SpecialCharTok{\^{}}\DecValTok{2}\NormalTok{))}
\FunctionTok{cat}\NormalTok{(}\StringTok{"RMSE:"}\NormalTok{, rmse, }\StringTok{"}\SpecialCharTok{\textbackslash{}n}\StringTok{"}\NormalTok{)}
\end{Highlighting}
\end{Shaded}

\begin{verbatim}
## RMSE: 0.09040613
\end{verbatim}

\begin{Shaded}
\begin{Highlighting}[]
\CommentTok{\# Plot}
\FunctionTok{ggplot}\NormalTok{() }\SpecialCharTok{+} 
  \FunctionTok{geom\_line}\NormalTok{(}\AttributeTok{data =}\NormalTok{ train, }\FunctionTok{aes}\NormalTok{(}\AttributeTok{x =}\NormalTok{ X, }\AttributeTok{y =}\NormalTok{ Y), }\AttributeTok{color =} \StringTok{"black"}\NormalTok{) }\SpecialCharTok{+}
  \FunctionTok{geom\_point}\NormalTok{(}\AttributeTok{data =}\NormalTok{ test, }\FunctionTok{aes}\NormalTok{(}\AttributeTok{x =}\NormalTok{ X, }\AttributeTok{y =}\NormalTok{ pred1), }\AttributeTok{color =} \StringTok{"blue"}\NormalTok{) }\SpecialCharTok{+}
  \FunctionTok{geom\_point}\NormalTok{(}\AttributeTok{data =}\NormalTok{ test, }\FunctionTok{aes}\NormalTok{(}\AttributeTok{x =}\NormalTok{ X, }\AttributeTok{y =}\NormalTok{ pred2), }\AttributeTok{color =} \StringTok{"red"}\NormalTok{) }\SpecialCharTok{+}
  \FunctionTok{geom\_point}\NormalTok{(}\AttributeTok{data =}\NormalTok{ test, }\FunctionTok{aes}\NormalTok{(}\AttributeTok{x =}\NormalTok{ X, }\AttributeTok{y =}\NormalTok{ pred3), }\AttributeTok{color =} \StringTok{"darkgreen"}\NormalTok{) }\SpecialCharTok{+}
  \FunctionTok{ggtitle}\NormalTok{(}\StringTok{"Prediction with 3 (blue) and 7 (red) hidden neurons. 100 iterations (dark green)"}\NormalTok{) }\SpecialCharTok{+}
  \FunctionTok{xlab}\NormalTok{(}\StringTok{"X"}\NormalTok{) }\SpecialCharTok{+} \FunctionTok{ylab}\NormalTok{(}\StringTok{"Y"}\NormalTok{)}
\end{Highlighting}
\end{Shaded}

\includegraphics{Exercise_8_files/figure-latex/unnamed-chunk-11-1.pdf}

The neural network failed to converge after 50 iterations. This was the
case for the model with three neurons in the hidden layer and with
seven, respectively. Therefore, we retrained the model with 100
iterations and reached convergence (seen in dark green in the plot
above).

\end{document}
